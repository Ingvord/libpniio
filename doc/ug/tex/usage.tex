%%%describing the usage of the library

In this section we will have a short look on how to make your code working the
\libpniio. The key to make using \libpniio\ simple is the usage of {\tt
pkg-config}. The rational behind the design descission to focus on {\tt
pkg-config} as the central element for build systems is simple: it works for
virtually all build systems. It is even available for Windows (though not very
often used). 

\section{From the command line}

\begin{minted}{bash}
    $> g++ -std=c++11 -otest test.cpp $(pkg-config --cflags --libs pniio)
\end{minted}
There are two important remarks we have to make here. The first is the {\tt
-std=c++11} after {\tt g++}. This tells the compiler to use the new C++11
standard. This option is absolutely required for the code to build. 
the {\tt pkg-config} command at the end of the command line includes all the
necessary compiler and linker flags to build and link the code.

\section{From within a Makefile}

{\tt pkg-config} can be used in a Makefile by putting the following at the top
of your Makefile
\begin{minted}{make}
CPPFLAGS=-O2 -g -std=c++11 $(shell pkg-config --cflags pniio)
LDFLAGS=$(shell pkg-config --libs pniio)
\end{minted}

\section{With CMake}

For {\tt cmake} the {\tt FindPkgConfig} module provides access to the
functionality of {\tt pkg-config}. The following snippet from a {\tt
CMakeLists.txt} file shows how to use it for \libpniio
\begin{minted}{cmake}
#load pkg-config package
include(FindPkgConfig)

#search for the pniio library 
pkg_search_module(PNIIO REQUIRED pniio)
link_directories(${PNIIO_LIBRARY_DIRS})
include_directories(${PNIIO_INCLUDE_DIRS})
add_definitions(${PNIIO_CFLAGS})

set(SOURCE ...)

add_executable(myprog ${SOURCE})
target_link_libraries(myprog ${PNIIO_LIBRARIES})
\end{minted}

