%%%-------------------------------------

The term legacy data refers to all non-Nexus file formats.
\libpnicore\ distinguishes between tow families of legacy formats 
\begin{itemize}
\item ASCII file where the content is entirely stored in human readable ASCII
characters
\item and binary data where the raw binary information is stored in a file.
\end{itemize}

%%%===========================================================================
\section{ASCII data}

\subsection{Lowlevel parser interface}

\libpniio\ provides a low level parser interface based on the \texttt{boost::spirit}
spirit framework. The major job of this interface is to provided save number
parsing. It supports all primitive data types provided by \libpnicore\ along
with \texttt{std::vector} begin a container of a primitive type.
For a detailed explanation about the low level parsers see
Appendix~\ref{appendix:parsers}. 

At the heart of the parser API is the \cpp{parser} class template. 
It takes one template parameter which is the primitive or container type 
to parse. To use the parser API just include \cpp{pni/io/parsers.hpp} in 
your source file. 

\subsubsection{Parsing primitive scalars}

A very simple example would be something like this
\begin{cppcode}
#include <iostream>
#include <pni/core/types.hpp>
#include <pni/io/parsers.hpp>

using namespace pni::core;
using namespace pni::io;

typedef parser<float64> float64_parser_type;

int main(int argc,char **argv)
{
    float64_parser_type p;

    float64 data = p("1.234");
    std::cout<<data<<std::endl;

    return 0;
}
\end{cppcode}
This example should be rather self explaining. 
When used with scalar values the parser template provides only a default 
constructor. No additional information is required to configure the 
parser code. 

Besides primitive types the \cpp{parser} template can also be used with 
the \cpp{value} type erasure. In this case the resulting parser matches 
either a \cpp{int64}, a \cpp{float64}, or a \cpp{complex64} type. Again 
no additional configuration at parser instantiation is required. 
For the ASCII representation of complex numbers see
Appendix~\ref{appendix:parsers}.

\subsubsection{Parsing a vector of primitives}

Besides single scalars the \cpp{parser} template can also be used with 
\cpp{std::vector} based containers where the element type should be one 
of the primitive types or a value. 
For this purpose a specialization of the \cpp{parser} template of the 
form
\begin{cppcode}
template<typename T> class parser<std::vector<T>> {...};
\end{cppcode}
is provided.
A particularly interesting choice as an element is the \cpp{value} type
erasure as it allows to parse a series of inhomogeneous types.
The following program
\begin{cppcode}
#include <iostream>
#include <vector>
#include <pni/core/types.hpp>
#include <pni/io/parsers.hpp>

using namespace pni::core;
using namespace pni::io;

typedef std::vector<value> record_type;
typedef parser<record_type> record_parser;

int main(int argc,char **argv)
{
    record_parser p;
    record_type data = p("1.234  12 1+I3.4");
    for(auto v: data)
        std::cout<<v.type_id()<<std::endl;

    return 0;
}
\end{cppcode}
would produce this output
\begin{verbatim}
FLOAT64
INT64
COMPLEX64
\end{verbatim}
When using the default constructor of the \cpp{parser} template with a 
container type the individual elements are considered to be separated by 
at least one blank. 
However the vector parser specialization
of the \cpp{parser} template provides three more additional constructors. 
The first, \cpp{parser(char del)} allows to use a custom delimiter symbol.
In the next example the \cpp{','} is used as a delimiter for the individual 
elements
\begin{cppcode}
record_parser p(',');
record_type data = p("1.234,12 , 1+I3.4");
\end{cppcode}
It is important to not that the delimiter symbol can be surrounded by an
arbitrary number of blanks. 
The second constructor provides the constructor with additional 
start and stop symbols. 
\begin{cppcode}
record_parser p('[',']');
record_type data = p("[1.234 12  1+I3.4]");
\end{cppcode}
However, the elements in the string are now again separated only by blanks. 
Full customization of the parser is provided by the third constructor which
allows the user to provide not only start and stop symbols but also a custom 
delimiter symbol
\begin{cppcode}
record_parser p('[',']',';');
record_type data = p("[1.234;12 ; 1+I3.4]");
\end{cppcode}

%%%===========================================================================
\section{Binary data}


