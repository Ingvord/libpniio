Sometimes one may wants to remove an object from a file. Though it is possible
to remove objects from a file one should use this operation with care. 
The reason for this is the due to a limitation of the HDF5 backend currently
used for storing data to disk. HDF5 cannot remove a data item physically from
disk. It only removes all links to it so that the object cannot be accessed any
more. 
This is the price one has to pay for random access. Physically removing the data
from disk would leave a hole in the HDF5 file. 
If one has to delete a large object, like a field with detector data, one should
run the \cpp{repack} command afterwards on the file which will all the deleted
objects by rewriting the file to disk.

In order to remove an object use the \cpp{remove} member function of \nxgroup. 
Let us suppose we want to delete the detector group from a \nexus-file. 
This could be done as follows
\begin{cppcode}
h5::nxgroup beamline = get_object(root,":NXentry/:NXinstrument");
beamline.remove("detector");
\end{cppcode}

The attribute manager provides a \cpp{remove} member function too. This can be
used to remove an attribute from a field or group. 
\begin{cppcode}
h5::nxfield field = get_object(root,....);
field.attributes.remove("temp_attribute");
\end{cppcode}

