%%% a basic introduction to Nexus

Today, data recorded during synchrotron experiments is typically stored in
individual binary image files and/or as flat ASCII files. 
Figure~\ref{fig:nxintro:old_fstree} shows the typicall directory structure of
such a setup. The ASCII file stores scalar data while the detector data is
stored in a separate directory as image files (here TIFF).
Such an approach leads to technical and organizational problems
\begin{enumerate}
\item when the number of image files grows large the performance of most file
systems degenerate 
\item to access data in an individual image file a new file handler has to be
created 
\item image and scalar data is stored in different files which increases the
managements efforts to keep related information aligned.
\end{enumerate}
\begin{figure}[tb]
    \centering
    \begin{minipage}[c]{0.4\linewidth}
    \centering
    \missingfigure{Need a figure of the old fs tree}
    %%\resizebox{\linewidth}{!}{\includegraphics{pics/old_fstree.pdf}}
    \end{minipage}
    \hspace{0.05\linewidth}
    \begin{minipage}[c]{0.5\linewidth}
    \caption{{\small A typical directory structure used at todays synchrotron
    experiments. Scalar data is stored in a single ASCII file while detector
    data is stored as individual image files in a separate directory.}}
    \label{fig:nxintro:old_fstree}
    \end{minipage}
\end{figure}

Nexus is a binary file format which attempts to solve all of these problems.
Nexus can keep scalar and multidimensional data within a single file and allows
to organize data in trees. Additional attributes can be attached to each object
in a file storing metadata which might be required for later analysis of the
data. It must be noted that Nexus is not a physical file format itself. It is
rather a set of rules how data must be organized within a particular format in
order to become a valid Nexus file. Currently the following physical file
formats are supported by the original Nexus API
\begin{itemize}
\item XML -- currently only used for file structure validation
\item HDF4 -- for historical reasons, should not be used for new data
\item HDF5 -- the current standard storage backend for Nexus files. 
\end{itemize}
One of the aims of \libpniio\ is to provide an abstraction layer between the
user and the storage backend. As \libpniio\ currently supports only HDF5 this is
rather artificial. However, \libpniio\ provides the architecture to include
other file formats to be used with Nexus too. 
There has been a lot of confusion what physical file format Nexus files are.
Many users think that Nexus has its own phyisical file format. This is in fact
not true. Just to avoid any further confusion fo the reader let me make this
clear once and for all
\begin{quote}
{\huge
{\color{red}Every Nexus file written by \libpniio\ is also a valid HDF5 file!}
\todo[caption={fix quote},inline]{This quote needs better formating. Maybe the
text should go into a box and the margins to the surrounding text must be
bigger}
}
\end{quote}

%%%===========================================================================
\section{The Nexus layer model}

\begin{figure}[tb]
    \centering
    \begin{minipage}[c]{0.4\linewidth}
    \centering
    \missingfigure{Need a figure of the Nexus layers}
    %%%\resizebox{\linewidth}{!}{\includegraphics{pics/nxlayers.pdf}}
    \end{minipage}
    \hspace{0.05\linewidth}
    \begin{minipage}[c]{0.5\linewidth}
    \caption{{\small Nexus can be considered to consist of three layers}}
    \label{fig:nxintro:layers}
    \end{minipage}
\end{figure}

\subsection{Layer 1 objects}

\begin{figure}[tb]
    \centering
    \begin{minipage}[c]{0.4\linewidth}
    \centering
    \missingfigure{Need a figure for the layer1}
    %%\resizebox{\linewidth}{!}{\includegraphics{pics/layer1.pdf}}
    \end{minipage}
    \hspace{0.05\linewidth}
    \begin{minipage}[c]{0.5\linewidth}
    \caption{{\small The basic objects of the first layer in the Nexus object
    model.}}
    \label{fig:nxintro:layer1}
    \end{minipage}
\end{figure}

\subsection{Layer 2 objects}

\subsection{Layer 3 objects}
