%%%describing the basic usage

This chapter deals with the basic interface provided by the layer 1 types
implemented in \libpniio. All types concerning Nexus reside in one of the
namespaces embedded in {\tt pni::io::nx}. The namespaces below this one 
indicate either a particular storage backend (currently only HDF5 is
implemented).

To use the Nexus part of the library just add 
\begin{minted}{cpp}
#include <pni/io/nx/nx.hpp>
\end{minted}
to your source file. 

%%%===========================================================================
\section{Dealing with files}
The most fundamental thing one wants to do is to create, open, and close files. 
A typical usage pattern for a file object would look like this
\begin{minted}{cpp}
#include <pni/io/nx/nx.hpp>

using namespace pni::io::nx;

int main(int argc,char **argv)
{
    h5::nxfile file = h5::nxfile::create_file("test.nxs");
    //... code omitted ...
    file.close();

    return 0;
}
\end{minted}
To create the file we use the {\tt create\_file} static member method of the
{\tt nxfile} class. The available signatures for this function are
\begin{minted}{cpp}
create_file(const string &n,bool ow=false, ssize_t ssize = 0);
create_file(const string &n,bool ow);
create_file(const string &n,ssize_t ssize);
\end{minted}
\todo{Add the missing create\_file signatures to the library}
The arguments have the following meaning
\begin{center}
\begin{tabular}{l|l}
argument & description \\
\hline\hline
{\tt const string \&n} & name of the file \\
\hline
{\tt bool ow} & overwrite an existing file \\
\hline
{\tt ssize\_t sszie} & split size \\
\hline
\end{tabular}
\end{center}
It is important to note the last argument {\tt ssize}. If this argument is
passed and not equal $0$ the HDF5 backend will use the split-driver to write the
data. In this case the file will be split into individual files of size {\tt
ssize} (in MByte). This requires the file name to be a valid C format string
containing an integer index as shown in the next example
\begin{minted}{cpp}
h5::nxfile file = h5::nxfile::create_file("test.%04i.nxs",1024);
\end{minted}
Here the split driver will produced files of $1$GByte size with file names
\begin{minted}{bash}
test.0001.nxs
test.0002.nxs
test.0003.nxs
...
\end{minted}
If a file already exist the {\tt open\_file} static member function of the {\tt
nxfile} should be used. 
Its signature is rather simple 
\begin{minted}{cpp}
open_file(const string &n,bool ro=true)
\end{minted}
where the first argument is again the name of the file to open. The second
optional argument determines whether the file will be opened read-only (the
default) or in read-write mode. 

The file type provides some more methods which should be mentioned here briefly 
\begin{center}
\begin{tabular}{l|l}
method & description\\
\hline\hline
{\tt flush()} & call this method after writing data to ensure that it is written
to disk \\
\hline
{\tt is\_valid()} & return true if the file is valid \\
\hline
{\tt close()} & close a file \\
\hline
{\tt is\_readonly()} & returns true if the file was opened in read-only mode \\
\hline
{\tt root()} & return the root group of the file \\
\hline
\end{tabular}
\end{center}
There are two important remarks to make about files. The first concerns the {\tt
close()} method. It is usually not necessary to call this method explicitly as
the file will be closed automatically when it looses scope. 
The second bears the {\tt flush()} method. This is a rather useful method and
should be called anytime a particular amount of data which can be considered
consistent has been submitted to the file for writing. The {\tt flush()} method 
tells the operating system to take over and write the data.


%%%===========================================================================
\section{Working with groups}

%%%===========================================================================
\section{Working with fields}

%%%===========================================================================
\section{Iterating groups}

%%%===========================================================================
\section{Working with attributes}

