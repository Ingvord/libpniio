%%%===========================================================================
\section{The mysterious \nxobject}\label{section:nxobject}

%%%===========================================================================
\section{Iterating groups}\label{section:group_iteration}

\subsection{Simple iteration}

The {\tt nxgroup} type provides an STL compliant iterator interface to iterate
over the direct children of a group. In this very simple example we loop over 
all entries stored in a file
\begin{cppcode}
h5::nxgroup root = f.root();

for(auto entry: root)
    std::cout<<entry.name()<<std::endl;
\end{cppcode}
Another interesting example would be to count all instances of {\tt NXdetector} 
within an instrument group
\begin{cppcode}
#include <pni/core/types.hpp>
#include <pni/io/nx/nx.hpp>
#include <pni/io/nx/algorithms.hpp>

using namespace pni::core;
using namespace pni::io::nx;

//predicate function
bool is_detector(const h5::nxobject &o)
{
    if(is_group(o)) return is_class(o,"NXdetector");
    else return false;
}

int main(int argc,char **argv)
{
    h5::nxfile file = h5::nxfile::open_file("test.nxs");
    h5::nxgroup instrument = get_object(root,"/:NXentry/:NXinstrument");

    size_t ndetectors = std::count_if(instrument.begin(),instrument.end(),
                                      is_detector);
    std::cout<<"Found "<<ndetectors<<" detectors!"<<std::endl;
    return 0;
}
\end{cppcode}

\subsection{Recursive iteration}

\subsection{Deleting items}

%%%===========================================================================
\section{Using algorithms}



