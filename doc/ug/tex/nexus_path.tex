%%% describing how to address Nexus objects

\newcommand{\fsection}{\textit{file section}}
\newcommand{\fsep}{\texttt{://}}
\newcommand{\osection}{\textit{object section}}
\newcommand{\asep}{\texttt{@}}
\newcommand{\asection}{\textit{attribute section}}
\newcommand{\nsection}{\textit{name section}}
\newcommand{\csection}{\textit{class section}}
\newcommand{\csep}{\texttt{:}}
\newcommand{\osep}{\texttt{/}}
\newcommand{\cgroup}{\texttt{.}}
\newcommand{\pgroup}{\texttt{..}}

%%%---------------------------------------------------------------------------
Objects within a \nexus-file can be referenced by a path. Though being very
similar to a Unix file system path, a \nexus-path provides much more
flexibility. It reflects one of the key features of \nexus: types. 
A \nexus-path can reference an object not only via its name (as HDF5 does) but
also by its type. Hence, under certain conditions, it is possible to construct
a path which is independent of the names choosen within a file.

%%%===========================================================================
\section{Introduction}
%%----------------------------------------------------------------------------
\begin{figure}[tb]
\centering
\begin{tikzpicture}
[cnode/.style = {rounded corners,draw=black,
                 minimum height = 0.75cm,
                 minimum width  = 1cm,
                 node distance = 0.5cm}]
\node (filenode) [cnode] {\texttt{filepath}};
\node (filesep)  [cnode,right = of filenode] {\texttt{://}};
\node (objnode)  [cnode,right = of filesep] {\texttt{name:class/:class/name}};
\node (attrsep)  [cnode,right = of objnode]  {\texttt{@}};
\node (attrnode) [cnode,right = of attrsep]  {\texttt{attribute name}};

\draw (filenode)--(filesep)--(objnode)--(attrsep)--(attrnode);

\draw[dashed] ($(filenode.west)+(-2mm,2mm)$) --
              node[below right=0.25cm and 0.75cm]{\fsection} 
              +(0,-1.5cm);
\draw[dashed] ($(filesep.east)+(2mm,2mm)$)  --
              node[below right=0.25cm and 1.5cm]{\osection}
              +(0,-1.5cm);
\draw[dashed] ($(attrsep.west)+(-2mm,2mm)$) --
              node[below right=0.25cm and 0.8cm]{\asection}
              +(0,-1.5cm);
\draw[dashed] ($(attrnode.east)+(2mm,2mm)$)  -- +(0,-1.5cm);

\end{tikzpicture}
\caption{\small\label{fig:path:structure}
The basic structure of a \nexus\ path as used by \libpniio. The \fsection\
stores the Unix path to the data file. The \osection\ the
path to the field or group within the file and the \asection\
holds the name of an attribute attached to the object referenced by the
previous path.
}
\end{figure}
%%----------------------------------------------------------------------------
\begin{figure}[tb]
\centering
\begin{minipage}[c]{0.6\linewidth}
\begin{tikzpicture}
[cnode/.style = {rounded corners,draw=black,
                 minimum height = 0.75cm,
                 minimum width  = 1cm,
                 node distance = 0.5cm}]
\node (name) [cnode,minimum width=3cm] {\texttt{name}};
\node (sep)  [cnode,right = of name] {\texttt{:}};
\node (class) [cnode,right = of sep,minimum width=3cm] {\texttt{class}};
\draw (name) -- (sep) -- (class);

\draw[dashed] ($(name.east)+(2mm,2mm)$) -- 
              node[below right= 0.25cm and 1.4cm]{\csection}
              node[below left =0.25cm and 0.5cm]{\nsection}
              +(0,-1.5cm);

\end{tikzpicture}
\end{minipage}
\hfill
\begin{minipage}[c]{0.39\linewidth}
\caption{\small\label{fig:path:object} 
Structure of the elements in the \emph{object section} of a \nexus-path.
}
\end{minipage}
\end{figure}
%%%===========================================================================
\subsection{The structure of a \nexus-path}

Figure~\ref{fig:path:structure} shows the principal structure of a \nexus-path
as used by \libpniio. Such a path comprises three major sections
\begin{inlinetab}{m{0.2\linewidth}m{0.65\linewidth}}
 \emph{file section}      & which references the \nexus-file on the file system.
 It must thus be a valid file system path on the operating system platform in
 use.  \\
 \emph{object section}    & describing the location of an object within the file \\
 \emph{attribute section} & referencing an attribute attached to the object 
 pointed to by the residual path. The attribute is identified by its name.
\end{inlinetab}
As shown in Fig.~\ref{fig:path:structure} the file and the object sections are 
separated by \fsep\ while the object and attribute sections use \asep\ as a
delimiter. Both, the file and the attribute section, are optional.
 The individual elements in the object section are separated by a
single \osep. Every element in the \osection\ is composed of two parts
(see Fig.~\ref{fig:path:object}): the \nsection\ and the \csection 
separated by a \csep. Whether or not the \nsection\ and/or the \csection\ must
be present in order to reference an element depends on the circumstances. 
There are three possible situations
\begin{inlinetab}{m{0.15\linewidth}m{0.75\linewidth}}
\emph{name:class} & this is a full identifier for a group. It determines its
name as well as its type. As fields have no type they cannot be referenced by
such an expression. \\
\emph{name} & if only the name is given the referenced object can be either a
group or a field. However, in the case of a field, this must be the last element
in the object path (as fields cannot have additional children). \\
\emph{:class} & if only the class is given the referenced object must be a
group. Denote the leading colon in this expression. It is necessary to
distinguish such an expression from a mere name.
\end{inlinetab}

%%%===========================================================================
\subsection{Some general path properties}

A path is considered as \emph{absolute} if its \osection\ starts at the root
group of the file. This is always  the case if 
\begin{itemize}
\item the \fsection\ of the path is not empty
\item or, if no \fsection\ is given, the \osection\ starts with a leading \osep.
\end{itemize}
The latter condition is equivalent to the convention used for Unix file system
paths while the former requires some explanation. 
If the \fsection\ is not empty the \osection\ has to be considered absolute
otherwise we would not know where to start searching for objects. 
If no \fsection\ is provided the path can also be a relative path which
references an object relative to an already given parent object.
\todo{maybe we should present some example here}

The \nexus-path implementation provided by \libpniio\ also understands 
\cgroup\ and \pgroup\ where the former one refers to the current group while 
the latter one to the parent group of the current group.

%%%===========================================================================
\subsection{Examples}
Let's have a look on some examples. The following path addresses the data field 
in the detector group of a file
\begin{verbatim}
/data/run/detector.nxs://entry/instrument/detector/data
\end{verbatim}
Here, the individual groups are referenced by their name in the object section 
of the path. Indeed, this path can be written in a more general way with 
\begin{verbatim}
/data/run/detector.nxs://:NXentry/:NXinstrument/:NXdetector/data
\end{verbatim}
where the parent groups of the \cpp{data} field are referenced implicitly via
their type.  This requires that only one instance of a particular type
(\cpp{:NXentry}, \cpp{:NXinstrument}, etc.)  exists in its parent group. In the
case that we have two detectors and each of them is stored as an instance of
\cpp{NXdetector} below the \cpp{NXinstrument} group, the name of the detector
must be provided explicitly 
\begin{verbatim}
/data/run/detector.nxs://:NXentry/:NXinstrument/det1:NXdetector/data
\end{verbatim}
The last group reference \cpp{det1:NXdetector} is the most precise 
description of a group instance. Not only does it determined the name 
of the group but also its type.  This example already shows one of the 
powers of \nexus. As long as only one instance of a particular type exists
within a group it can be identified by its type rather than by its name. 
In many situations it is thus possible to generate paths which are virtually
independent of all object names (in fact only the fields must be named as they
have no type).

All path examples until now represented an absolute path (a path with a leading
\fsection). In many situations no file must be specified. A typical application
for paths without \fsection\ would be program where an object should be
referenced by a path relative to a given parent object. 
The path in the next example references the data field of the detector 
relative to the top level instance of \cpp{NXentry}
\begin{verbatim}
:NXinstrument/detector/data 
\end{verbatim}
In order to make a path without a \fsection\ \emph{absolute}, it must 
start with a leading \cpp{/} as in the next example
\begin{verbatim}
/:NXentry/:NXinstrument/pilatus/data
\end{verbatim}

In order to reference the root group of a file one can either use 
\begin{verbatim}
/
\end{verbatim}
a single \osep\ or, in case of a file section
\begin{verbatim}
/data/run/detector.nxs://
\end{verbatim}
where the trailing \fsep\ denotes the root group. In case of an absolute path
the root group is always included in the path object (as will be shown later). 

Finally an application for  \pgroup\ should be discussed. Lets assume that the
current parent is the detector group and we want to address the diameter field
of an instance of \cpp{NXpnihole} located one level above. We could do this with
\begin{verbatim}
../:NXpnihole/diameter
\end{verbatim}
where \pgroup\ indicates that one should move to the parent group of the 
current one.

%%%===========================================================================
\section{The \nxpath\ type}

In C++ a \nexus-path is represented by an instance of \nxpath. 
To use \nxpath\ and its utility functions the appropriate header file must be 
included 
\begin{cppcode}
#include <pni/io/nx/nxpath.hpp>
\end{cppcode}
\nxpath\ is an
iterable over the elements of the \osection\ of a \nexus-path.  
The optional \emph{file-} and \asection\ can be accessed via getter and setter
methods like this
\begin{cppcode}
nxpath path = ...;
path.filename("/data/run/detector.nxs"); //set file section
std::cout<<path.filename()<<std::endl;   //retrieve file section
\end{cppcode}
and analogously for the \asection
\begin{cppcode}
nxpath path = ...;
path.attribute("units");              //set attribute section
std::cout<<path.units()<<std::endl;   //retrieve attribute section
\end{cppcode}

The elements of the \osection\ are stored as instances of 
\cpp{std::pair<string,string>} where the first element of the pair holds the
name of the element and the second the class (if available). 
\nxpath\ provides an alias to the element type via the public member type
\cpp{nxpath::element\_type}.
Technically, \nxpath\ is a thin wrapper around a list of such
\cpp{element\_type} (although not all the list functionality is exported).
Consult the API documentation for a detailed description of \nxpath's interface.


%%%---------------------------------------------------------------------------
\subsection{Path construction}

Though the \nxpath\ type has a constructor one would typically construct 
a path from a string using the \cpp{from\_string} static member method
\begin{cppcode}
nxpath path = nxpath::from_string("/:NXentry/:NXinstrument/pilatus");
\end{cppcode}
\cpp{from\_string} has also a static counterpart method \cpp{to\_string} which 
converts a path instance to its string representation.
\begin{cppcode}
nxpath path = ....;
std::cout<<nxpath::to_string(path)<<std::endl;
\end{cppcode}

%%%---------------------------------------------------------------------------
\subsection{Path iteration}

\nxpath\ provides an STL compliant iterator interface which allows easy
iteration over all elements in the \osection\ of the path. Consider the 
following example
\begin{cppcode}
nxpath p = nxpath::from_string("/:NXentry/:NXinstrument/pilatus/data");

for(auto e:p)
    std::cout<<"name: "<<e.first<<"\t type:"<<e.second<<std::endl;
\end{cppcode}
which would yield the output
\begin{verbatim}
name: /       type: NXroot
name:         type: NXentry
name:         type: NXinstrument
name: pilatus type: 
name: data    type: 
\end{verbatim}
As we can see from the above example: the first member of the
\cpp{std::pair<string,string>} stored in the object section list is the 
name of an object while the second is its type. In the case of a field 
only the first (name) element will be set (a field does not have a 
particular type). 
The number of elements in the \osection\ of \nxpath\ can be obtained via the 
\cpp{size} member function (which is the same as for any other STL container).

%%%---------------------------------------------------------------------------
\subsection{Push and pop on object}

Elements of the \osection\ of the path can be added using the \cpp{push\_back}
and \cpp{push\_front} member functions. 
\begin{cppcode}
nxpath p = nxpath::from_string(":NXinstrument");
std::cout<<p<<std::endl; // output: :NXinstrument

p.push_back(object_element("","NXdetector"));
std::cout<<p<<std::endl; // output: :NXinstrument/:NXdetector

p.push_front(object_element("","NXentry"));
std::cout<<p<<std::endl; // output: :NXentry/:NXinstrument/:NXdetector
\end{cppcode}
Like other STL containers \nxpath\ also provides the \cpp{front}, \cpp{back}, 
\cpp{pop\_front}, and \cpp{pop\_back} member functions which have the 
standard STL behavior. 
\begin{cppcode}
nxpath p = nxpath::from_string(":NXentry/:NXinstrument/:NXdetector");

//get front and back elements from the object section
nxpath::element_type entry = p.front();
nxpath::element_type detector = p.back();

std::cout<<p<<std::endl; // output: :NXentry/:NXinstrument/:NXdetector

//remove front and back objects from the object section
p.pop_front();
p.pop_back();

std::cout<<p<<std::endl; // output: :NXinstrument

\end{cppcode}




%%%===========================================================================
\section{Utility functions}

\subsection{Element utilities}

There are a couple of utility functions available to work with the elements 
stored in the \osection\ of the path.
One important function is the \cpp{object\_element} function which 
creates a single element for the \osection\ of a path. This is particularly 
useful in connection with the \cpp{push\_back} and \cpp{push\_front} member 
functions of \nxpath. 
If for instance one wants to append a detector group to the object section
we could use
\begin{cppcode}
nxpath p = ...;
p.push_back(object_element("detector","NXdector"));
\end{cppcode}
\cpp{object\_element} takes two arguments: the first is the name of the object
while the second its type (only relevant for groups). If both are empty strings
and exception will be thrown.

Furthermore there are some functions for querying the basic properties of an 
element instance. Each of these functions returns a boolean value and takes
an instance of \cpp{nxpath::element\_type} as its only argument.
\begin{inlinetab}{m{0.2\linewidth}m{0.7\linewidth}}
\cpp{is\_root\_element} & returns true if the element references the root group
(with name \cpp{/} and type \cpp{NXroot})\\
\cpp{is\_complete}      & return true if the element has a non-empty name and 
                          type\\
\cpp{has\_name}         & return true if the element has a non-empty name\\
\cpp{has\_class}        & return true if the element has a non-empty type
\end{inlinetab}

\subsection{\nxpath\ utilities}

Three inquiry functions exist for \nxpath. Each of them returns a
boolean and takes as their single argument a reference to an instance of \nxpath
\begin{inlinetab}{m{0.3\linewidth}m{0.6\linewidth}}
\cpp{is\_absolute}            & returns \cpp{true} if the path is an absolute path\\
\cpp{has\_file\_section}      & returns \cpp{true} if the path has a non-empty file 
                                section\\
\cpp{has\_attribute\_section} & returns \cpp{true} if the path has a non-empty 
                                attribute section \\
\cpp{is\_empty}                & returns \cpp{true} if a path has neither a
\fsection, an \asection, and an \osection. This situation would be equivalent to
a default constructed path object.
\end{inlinetab}
The \cpp{split\_path} function divides an \nxpath\ into two partial paths 
at a user defined position. 
\begin{cppcode}
string s = "test.nxs://:NXentry/:NXinstrument/detector@NX_class";
nxpath p = nxpath::from_string(s);
nxpath instrument_path,detector_path;
split_path(p,3,instrument_path,detector_path);

// output: test.nxs://:NXentry/:NXinstrument
std::cout<<instrument_path<<std::endl; 
// output: detector@NX_class
std::cout<<detector_path<<std::endl;   
\end{cppcode}
The second argument to \cpp{split\_path} is the position where to perform the
split. It is the index of the first element for the second path.
To chop of the \fsection\ from a path one could use the following code
\begin{cppcode}
string s = "test.nxs://:NXentry/:NXinstrument/detector@NX_class";
nxpath p = nxpath::from_string(s);
nxpath instrument_path,detector_path;
split_path(p,0,instrument_path,detector_path);

// output: test.nxs
std::cout<<instrument_path<<std::endl;
// output: /:NXentry/:NXinstrument/detector@NX_class
std::cout<<detector_path<<std::endl;   
\end{cppcode}
Two paths can be joined using the \cpp{join()} function. 
\begin{cppcode}
nxpath a = nxpath::from_string("file.nxs://:NXentry/:NXinstrument");
nxpath b = nxpath::from_string("pilatus300k:NXdetector/data");
nxpath c = join(a,b);
std::cout<<c<<std::endl;

//would output
//file.nxs://:NXentry/:NXinstrument/pilatus300k:NXdetector/data"
\end{cppcode}
There are several restrictions to the two path arguments \cpp{a} and \cpp{b}
passed to the \cpp{join()} funtion
\begin{itemize}
\item \cpp{a} must not have an \asection
\item \cpp{b} must not have a \fsection
\item \cpp{b} must not be an absolute path.
\end{itemize}
If any of these restrictions are violated \cpp{join()} throws
\cpp{value\_error}. There are additional special conditions which should be
taken into account and where the above rules do not apply
\begin{inlinetab}{m{0.2\linewidth}m{0.1\linewidth}m{0.3\linewidth}}
\cpp{a} empty, \cpp{b} not &$\rightarrow$ & return \cpp{b} unchanged \\
\cpp{b} empty, \cpp{a} not &$\rightarrow$ & return \cpp{a} unchanged \\
\cpp{a} and \cpp{b} empty  &$\rightarrow$ & return an empty path object
\end{inlinetab}

%%%===========================================================================
\section{The grammar of a Nexus path}
Lets first have a look on the grammar of a Nexus path in
EBNF\footnote{EBNF=Extended Backus Naur Form}
\begin{verbatim}
file_path   = {all characters allowed by the plattform to describe a path}

(* definition of character sets*)
valid_char  = "_" | "a-z" | "A-Z" | "0-9";
whitespace  = " " | "\n" | "\r";

(*definition of required terminal symbols*)
class_seperator  = ":";
object_seperator = "/";
current_group    = ".";
parent_group     = "..";

(*a nexus ID must not be empty*)
nexus_id    = valid_char,{valid_char}; 
nexus_name  = nexus_id,(class_seperator|group_separtor|whitespace);
nexus_group = group_seperator,nexus_id,[group_seperator|whitespace];

(*
 the first part is the object name, the second the group class if the 
 object is a group
*)
object_id   =   nexus_name    
              | nexus_name,nexus_group 
              | nexus_group   
              | current_group 
              | parent_Gruop  
                                             
object_path ::= ["/"],object_id,{"/",object_id};
nexus_path  ::= [file_path,"://"],object_path,["@",nexus_attr];
\end{verbatim}

The {\tt file\_path} is platform dependent which makes it difficult to determine
which characters would be allowed in a path. Thus we leave this open to and
separate the file path from everything else by a {\tt ://} string germinal.
{\tt nexus\_id} describes a repetition of a set of characters allowed in Nexus
names (for groups, fields, attributes, and classes). It is much more restrictive
as for the filename.
