%%% describing how to address Nexus objects

Nexus objects are addressed by a path. In \libpniio\ the path of a Nexus object
consists of three parts
\begin{itemize}
\item the file section
\item the object section describing the location of an object within the file
\item and an optional attribute section.
\end{itemize}
The different sections are organized as follows
\begin{verbatim}
file://object_path@attribute
\end{verbatim}

\section{The grammar of a Nexus path}
Lets first have a look on the grammar of a Nexus path in
EBNF\footnote{EBNF=Extended Backus Naur Form}
\begin{verbatim}
file_path   = {all characters allowed by the plattform to describe a path}

(* definition of character sets*)
valid_char  = "_" | "a-z" | "A-Z" | "0-9";
whitespace  = " " | "\n" | "\r";

(*definition of required terminal symbols*)
class_seperator  = ":";
object_seperator = "/";
current_group    = ".";
parent_group     = "..";

(*a nexus ID must not be empty*)
nexus_id    = valid_char,{valid_char}; 
nexus_name  = nexus_id,(class_seperator|group_separtor|whitespace);
nexus_group = group_seperator,nexus_id,[group_seperator|whitespace];

(*
 the first part is the object name, the second the group class if the 
 object is a group
*)
object_id   =   nexus_name    
              | nexus_name,nexus_group 
              | nexus_group   
              | current_group 
              | parent_Gruop  
                                             
object_path ::= ["/"],object_id,{"/",object_id};
nexus_path  ::= [file_path,"://"],object_path,["@",nexus_attr];
\end{verbatim}

The {\tt file\_path} is platform dependent which makes it difficult to determine
which characters would be allowed in a path. Thus we leave this open to and
separate the file path from everything else by a {\tt ://} string germinal.
{\tt nexus\_id} describes a repetition of a set of characters allowed in Nexus
names (for groups, fields, attributes, and classes). It is much more restrictive
as for the filename.
