%%low level objects in the library

\pninx s' API is quite minimalistic. From a users point of view only 
three classes are required: {\tt NXFile}, {\tt NXGroup}, and {\tt NXField}.
This are the basic components required to create a valid Nexus file and 
fill it with data. Each of this classes will be presented in a separate 
section of this chapter. 
Because using links takes a little more explanation this topic has 
its own section in at the very end of this chapter.

To start using \pninx\ you only need to include a single header file 
and maybe set a proper namespace
\begin{minted}[linenos=true]{c++}
#include<pni/nx/NX.hpp>

using namespace pni::nx::h5;
\end{minted}

The namespace selects which implementation to use. Actually only {\tt h5} for
HDF5 is supported.

\section{NXFile}\label{section:nxfile}
\nxfile\ is the root to create all other data structures. 
\nxfile\ is a descendant of \nxgroup\ and thus provides all the functionality of
its base class. Here we will only discuss features unique to the \nxfile\ class.
The next chapter deals with \nxgroup.
There is not too much specific with a file you can do except create or open one.
The former procedure is shown in this example:
%%\lstinputlisting{../examples/c++/nxfile_ex1.cpp}
\inputminted[linenos=true]{c++}{../examples/c++/nxfile_ex1.cpp}

The file is created in line $9$. Unlike other objects a file is not 
created by a constructor but by static factory methods of \nxfile.
The method {\tt NXFile::create\_file} takes three arguments: 
\begin{enumerate}
    \item the name of the file to create (here {\tt file\_ex1.h5})
    \item the overwrite flag which causes an already existing file of same name
    to be overwriten (as shown in the above example)
    \item the split size (set this to $0$ for now).
\end{enumerate}
To open a file use the factory method {\tt NXFile::open\_file} like this
\begin{minted}{c++}
    NXFile file = NXFile::open_file("file_ex1.h5",true);
\end{minted}
The only two arguments are
\begin{enumerate}
    \item the name of the file
    \item the read-only flag which causes the file to be opened in read-only
    mode of true.
\end{enumerate}




\section{NXGroup}\label{section:nxgroup}
%%%documentation for NXGroup

Although residing below {\tt NXFile} in an files' object hierarchy, {\tt
NXGroup} is most probably the most powerful object of the three basic 
objects. 
Indeed {\tt NXFile} is a descandent of this class and provides all its
features in addition to the file specific methods shown in
section~\ref{section:nxfile}. Groups are containers to hold {\tt NXField}
objects or other groups. The next example shows how to create group objects.
\lstinputlisting{../examples/c++/nxgroup_ex1.cpp}
As one can see from this example, an {\tt NXFile} objects behaves pretty 
much like a group object. In lines 15 and 16 a group is created 
using a file and a group object. You cannot create a valid group object
using its constructor. Instead you have to use the {\tt createGroup} methods
provided by {\tt NXGroup}/{\tt NXFile}.
In Nexus groups usually have a type (class). The type of a group is 
stored in a string attribute name {\tt NX\_class} of the group object. 
To make life easier you can pass the type of a group directly at 
group direction to the {\tt createGroup} method as shown in line 17. 
Not existing intermediate groups are created automatically as can be 
seen in line 18.
Groups can be opened using the {\tt openGroup} method (see lines 21 and 22).
That's basically everything you need to know about groups for now.
Lets continue with the really interesting data holding object - {\tt NXField}. 





\section{NXField}\label{section:nxfield}
%%%NXField documentation

In the Nexus world the payload (your real data) is stored in fields which 
are represented in this library by objects of class {\tt NXField}.
Like groups, fields cannot be instantiated by their constructor but are 
rather created using factory methods provided by all classes derived 
from  {\tt NXGroup}. One word of caution: a prerequisit for understanding 
this section is a fundamental knowledge of the data objects provided 
by {\tt libpniutils}. If you are not please consult the {\tt libpniutils}
users guide first.
One can distinguish between two basic types of fields: string-fields
and numeric-fields. The former hold string data while the later 
holds numbers.   

\subsection{Creating fields}

The first step to use fields is to create them. Field creation for 
string and numeric fields is shown in this next example. 
\lstinputlisting{../examples/c++/nxfield_ex1.cpp}.
As we have seen in section~\ref{section:nxfield_design} the handling 
of data in \nxfield\ depends on the class of data that should be stored 




\subsection{Reading and writing numericaldata}
\label{section:nxfield_numeric_io}

\subsection{Reading and writing string data}
\label{section:nxfield_string_io}

\subsection{Reading and writing binary data}
\label{section:nxfield_binary_io}


\section{Links}\label{section:links}

\section{A full example}\label{section:full_example}
