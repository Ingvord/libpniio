%%%documentation for NXGroup

Although residing below {\tt NXFile} in an files' object hierarchy, {\tt
NXGroup} is most probably the most powerful object of the three basic 
objects. 
Indeed {\tt NXFile} is a descandent of this class and provides all its
features in addition to the file specific methods shown in
section~\ref{section:nxfile}. Groups are containers to hold {\tt NXField}
objects or other groups. The next example shows how to create group objects.
\lstinputlisting{../examples/c++/nxgroup_ex1.cpp}
As one can see from this example, an {\tt NXFile} objects behaves pretty 
much like a group object. In lines 15 and 16 a group is created 
using a file and a group object. You cannot create a valid group object
using its constructor. Instead you have to use the {\tt createGroup} methods
provided by {\tt NXGroup}/{\tt NXFile}.
In Nexus groups usually have a type (class). The type of a group is 
stored in a string attribute name {\tt NX\_class} of the group object. 
To make life easier you can pass the type of a group directly at 
group direction to the {\tt createGroup} method as shown in line 17. 
Not existing intermediate groups are created automatically as can be 
seen in line 18.
Groups can be opened using the {\tt openGroup} method (see lines 21 and 22).
That's basically everything you need to know about groups for now.
Lets continue with the really interesting data holding object - {\tt NXField}. 



