%%%administrators guide for libpniutils

\documentclass[a4paper,twoside]{book}
\usepackage{a4wide,amsmath,graphics}
\usepackage{listings}
\usepackage{float}
\usepackage{hyperref}
\usepackage{minted}
\author{Eugen Wintersberger}
\title{{\Huge libpninx\\ Users guide}}

%%setup the listings package
\lstset{language=C++}
\lstdefinestyle{numbers}{numbers=left,stepnumber=1,numberstyle=\small}
\lstset{style=numbers}
\lstset{captionpos=b,frame=lines,float=tb}
\lstset{floatplacement=tb}
\lstset{basicstyle=\small}
\lstset{tabsize=2}

%%%setup for hyperref
\hypersetup{pdftitle=libpninx Users Guide,
                pdfborder=0 0 0,
                colorlinks=false}

%%define some custom commands
\newcommand{\pninx}{{\tt libpninx}}
\newcommand{\nxfield}{{\tt NXField}}
\newcommand{\nxobject}{{\tt NXObject}}
\newcommand{\nxfile}{{\tt NXFile}}
\newcommand{\nxgroup}{{\tt NXGroup}}
\newcommand{\nxselection}{{\tt NXSelection}}
\newcommand{\nxattribute}{{\tt NXAttribute}}
\newcommand{\nxnumericfield}{{\tt NXNumericField}}
\newcommand{\nxstringfield}{{\tt NXStringField}}
\newcommand{\nxbinaryfield}{{\tt NXBinaryField}}
\newcommand{\arrayt}{{\tt Array<>}}
\newcommand{\scalart}{{\tt Scalar<>}}
\newcommand{\pniutils}{{\tt libpniutils}}

%%%setup custom chapter header 
\floatstyle{ruled}
\newfloat{mylistings}{htb}{lop}[chapter]

\makeatletter
\renewcommand{\@makechapterhead}[1]{%
\vspace*{50 pt}%
{\setlength{\parindent}{0pt} \raggedright \normalfont
\bfseries\Huge
\ifnum \value{secnumdepth}>1 
   \if@mainmatter\thechapter.\ \fi%
\fi
#1\par\nobreak\vspace{40 pt}}}
\makeatother
%%------------------------------------------

\begin{document}
\maketitle
\tableofcontents

\chapter{How to read}\label{chapter:how_to_read}

\chapter{Design overview}\label{chapter:design_overview}
%%%design overview over libpninx

\chapter{Basic usage - the most important classes}\label{chapter:basic_usage}
%%low level objects in the library

The API is quite minimalistic. From a users point of view only 
three classes are required: {\tt NXFile}, {\tt NXGroup}, and {\tt NXField}.
This are the basic components required to create a valid Nexus file and 
fill it with data. Each of this classes will be presented in a separate 
section of this chapter. 
Because using links takes a little more explanation this topic has 
its own section in at the very end of this chapter.

\section{NXFile}
\nxfile\ is the root to create all other data structures. 
\nxfile\ is a descendant of \nxgroup\ and thus provides all the functionality of
its base class. Here we will only discuss features unique to the \nxfile\ class.
The next chapter deals with \nxgroup.
There is not too much specific with a file you can do except create or open one.
The former procedure is shown in this example:
%%\lstinputlisting{../examples/c++/nxfile_ex1.cpp}
\inputminted[linenos=true]{c++}{../examples/c++/nxfile_ex1.cpp}

The file is created in line $9$. Unlike other objects a file is not 
created by a constructor but by static factory methods of \nxfile.
The method {\tt NXFile::create\_file} takes three arguments: 
\begin{enumerate}
    \item the name of the file to create (here {\tt file\_ex1.h5})
    \item the overwrite flag which causes an already existing file of same name
    to be overwriten (as shown in the above example)
    \item the split size (set this to $0$ for now).
\end{enumerate}
To open a file use the factory method {\tt NXFile::open\_file} like this
\begin{minted}{c++}
    NXFile file = NXFile::open_file("file_ex1.h5",true);
\end{minted}
The only two arguments are
\begin{enumerate}
    \item the name of the file
    \item the read-only flag which causes the file to be opened in read-only
    mode of true.
\end{enumerate}




\section{NXGroup}

\section{NXField}

\section{Link creation}

\chapter{Library configuration}
TO BE WRITTEN

\newpage
\addcontentsline{toc}{chapter}{Bibliography}
\bibliographystyle{ieeetr}
\bibliography{users-guide}

\end{document}
